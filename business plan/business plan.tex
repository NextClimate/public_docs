\documentclass[12pt]{article}
\usepackage{graphicx}
\begin{document}

\pagestyle{empty}
\renewcommand{\descriptionlabel}[1]{\hspace{\labelsep}\emph{#1}}
\begin{center}
\includegraphics{/Users/rwpinder/Projects/next.climate/src/gae/nextclimate/skeleton/images/nextclimate_logo_transparent.png}
\Large{BUSINESS PLAN}
\end{center}
\noindent


\subsubsection*{MISSION STATEMENT}

Our goal is to inspire people and communities to take action on
climate change. Our unique approach is to connect climate-conscious
consumers with cost-saving methods for improving their energy
efficiency.

\subsubsection*{BUSINESS CASE}

Our strategy is to use the internet to raise awareness about climate
change and to connect concerned citizens and communities with energy
efficiency solutions.

While nearly all people are aware of climate change, the implications
are not thoroughly understood. Most climate science is presented to
the public in abstract terms: global temperature, sea level rise, ice
cap melt, etc. The implications of these changes, on a local level,
have not been well communicated. People are left wondering -- what
does climate change mean for me? This means there is a large gap
between the severity of the impacts of climate change and people's
motivation to act. Our goal is to increase awareness of the local
impacts of climate change as a way to inspire communities to take
action.

Taking action on climate change is usually quantified in terms of
reducing your carbon footprint. The carbon footprint is the sum total
greenhouse gas emissions that can be attributed to your actions. For
most people and organizations, the largest part of their carbon
footprint is energy from the combustion of fossil fuels. Fortunately
there are many options to improve energy efficiency -- ways to reduce
greenhouse gases while at the same time saving money due to avoided
energy costs.

However, when people are motivated to reduce their carbon footprint,
they are faced with a wide array of confusing options. Firstly, the
most effective options vary from place to place. While home roof-top
solar panels are likely to be cost-effective in Arizona, they may be
of less value in cloudy Seattle. Extra insulation would likely be a
better option. Second, solutions designed for the local environment
are likely to be more cost-effective than nationally advertised
options. Consider some of the most well-communicated ways to reduce
your carbon footprint: compact fluorescent lightbulbs (CFLs) and hybrid
electric cars. CFLs are helpful, but only a small part of total energy
use. Hybrid electric cars are more efficient, but are costly compared
to conventional fuel-efficient cars. Accordingly, when people are
interested in reducing their carbon footprint, they don't have the
information they need. There are many, locally available energy
efficiency options that can both save money, due to lower energy
costs, and maximize reductions in greenhouse gases.


In summary, the twin problems we seek to solve are 
\begin{enumerate}
\item people are not sufficiently concerned about climate change,
  because they are not aware of the consequences to their local area
\item people make inefficient decisions when trying to mitigate
  climate change, because they are not aware of options that are most
  cost-effective for their local area
\end{enumerate}

Our hypothesis is that if people were more informed of the local-level
risks of climate change, and were immediately connected with services
and products in their community, they would make choices to improve
their energy-efficiency, save money, and reduce their carbon
footprint.

\subsubsection*{BUSINESS STRATEGY}

To solve these problems in an economically sustainable way, we will
develop web tools to connect people with energy efficiency services
and solutions in their local area.

This is a two part effort.

The first is to develop a web page that allows people to really
explore climate change information on a local level. Users will enter
their zip code, and be shown compelling graphics displaying the risks
and vulnerabilities due to climate change in their area. This content
will be sourced from the international scientific community as well as
developed by our research team. Furthermore, we will develop tools
that will easily allow the scientific community to share their climate
change findings, at a local scale, on our website. Users will be
provided the opportunity to comment and share this information with
their friends and colleagues.

When confronted with local-scale risks of climate change, users will
be encouraged to click a button labeled ``Act Now!''. This brings the
user to the second phase.

Based on the user's zipcode, they are shown a list of
energy-efficiency technologies, cost-saving techniques, and tips that
are relevant for that user's zipcode. The user is invited to provide
information to better refine the options.

At this point, we have identified a user, in a particular location,
that is interested in making a purchase to improve energy efficiency
and save energy expenses. To provide a sustainable stream of income,
we will sell sponsorships, on a zip code / metro area basis, for
companies and products that provide such services. We will rigorously
screen potential sponsors and only present products and services that
are effective, robust, and credible. A fraction of the presented
options will be reserved for public service information -- energy
efficiency tips that have no obvious paying sponsor.


\subsubsection*{STATEMENT OF VALUE ADDED}

We are adding value by funnelling climate-conscious consumers to
credible, local providers of energy-efficiency products and services.
Our hypothesis is that green technology and service companies are
willing to pay to sponsor this activity.

\subsubsection*{BUDGET}

\begin{tabular}{llr}
\emph{Year 1} &               &         \\
              & Revenues      &   (\$)  \\
              &  Grants       &  5,000  \\
              & Web sponsors  &    200  \\
              &               &         \\
              & Expenses      &   (\$)  \\
              &  Computing    &  5,200  \\
              &               &         \\
              &               &         \\
\emph{Year 2} &               &        \\
              & Revenues      &   (\$)  \\
              &  Grants       &  5,000  \\
              & Web sponsors  &  4,000  \\
              &               &         \\
              & Expenses      &   (\$)  \\
              &  Computing    &  5,000  \\
              &  Office space &  4,000  \\
              &               &         \\
              &               &         \\
\emph{Year 3} &               &        \\
              & Revenues      &   (\$)  \\
              & Web sponsors  &  10,000 \\
              &               &         \\
              & Expenses      &   (\$)  \\
              &  Computing    &  5,000  \\
              &  Office space &  5,000  \\
\end{tabular}


\end{document}